\documentclass[border=0pt]{standalone}
\usepackage[siunitx, RPvoltages]{circuitikz}
\begin{document}
	\begin{circuitikz}
		% PFET at the top
		\draw (0,0.5) node[pfetd, anchor=D, label=right:{\scriptsize$P_1$}](P){}; % Drain anchor at (0,0.5)
		
		% NFET at the bottom, aligned so its drain is connected to the PFET's drain
		\draw (0,0) node[nfetd, anchor=D, label=right:{\scriptsize$N_1$}](N){};
		
		% Connect the drain of PFET to drain of NFET (output node Y centered)
		\draw (P.D) -- (N.D);
		\node[circle, fill=white, draw=black, inner sep=1pt, label=right:$Y$] at (0,0.25) {}; % Y zentriert, kleiner weißer Kreis
		
		% Connect the gates together (input A centered)
		\coordinate (G) at (-1,0.25);              
		\draw (P.G) -| (G) |- (N.G);
		\node[circle, fill=white, draw=black, inner sep=1pt, label=left:$A$] at (G) {};       % A zentriert, kleiner weißer Kreis
		
		% Connect the source of PFET to VDD
		\draw (P.S) -- ++(0,0) node[circle, fill=white, draw=black, inner sep=1pt, label=right:$U$]{};
		
		% Connect the source of NFET to ground
		\draw (N.S) -- ++(0,0.2) node[ground]{};
	\end{circuitikz}
\end{document}